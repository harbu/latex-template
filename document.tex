\documentclass[a4paper,12pt]{article}

\usepackage[T1]{fontenc}        % cleaner font
\usepackage[utf8]{inputenc}     % UTF-8 support
\usepackage{amsmath}            % more math symbols
\usepackage{amssymb}            % allow math symbols of form \mathbb{...}
\usepackage{graphicx}           % enhanced graphics support
\usepackage{epstopdf}           % automagically turn eps to pdf, for gnuplot
\usepackage{subcaption}         % subfigure support
\usepackage[dvipsnames]{xcolor} % syntax coloring support
\usepackage{listings}           % programming language support
\usepackage{algorithmic}        % pseudocode support
\usepackage{algorithm}
\usepackage{cancel}             % cancel out terms in division
\usepackage{parskip}            % enable spacing between paragraphs
\usepackage{cases}              % enable math function definitions with cases
\usepackage{titling}            % allow adjustment of document title
\usepackage{fullpage}           % 1 inch margins
\usepackage{bm}                 % bold math fonts for vectors and stuff
\usepackage{nicefrac}           % Nicer looking fractions
\usepackage{floatflt}           % Allows text to float around figures / tables
\usepackage{biblatex}           % Support for smarter bibliographies

%\usepackage[finnish]{babel}    % use Finnish for hyphenation
%\usepackage{hyperref}          % make clickable urls
%\usepackage{color}             % colored text?

% Adjust title vertical position
\setlength{\droptitle}{-2cm}

% Set up syntax highlighting for programming languages
\lstloadlanguages{Ruby}
\lstset{%
basicstyle=\ttfamily\bfseries\footnotesize,
commentstyle = \ttfamily\color{orange},
keywordstyle=\ttfamily\color{blue},
stringstyle=\color{red},
showstringspaces=false,
frame=trbl,
}

% Custom probability macros
\def\ci{\perp\!\!\!\perp}              % Independence symbol
\newcommand{\jpr}[2]{P(#1 \, , \, #2)} % Joint probability
\newcommand{\cpr}[2]{P(#1 \, | \, #2)} % Conditional probability

% Custom topological macros
\newcommand\opn{\mathrel{\ooalign{$\subset$\cr        % open set \opn
  \hidewidth\hbox{$\circ\mkern.5mu$}\cr}}}
\newcommand\cls{\mathrel{\ooalign{$\subset$\cr        % closed set \cls
  \hidewidth\raise.225ex\hbox{$\text{{\scriptsize c}}\mkern2mu$}\cr}}}

% Custom handy macros for tuples and sets in maths
\newcommand{\setof}[1]{\ensuremath{\left \{ #1 \right \}}}
\newcommand{\tuple}[1]{\ensuremath{\left \langle #1 \right \rangle }}

% Finnish-style limits after integration symbol (by Martti Nikunen)
\newcommand{\vii}{\mathop{\Big/}}
\newcommand{\viiva}[2]{\vii\limits_{\!\!\!\!{#1}}^{\>\,{#2}}}

% Configure bibliography database
\bibliography{sources.bib}

\begin{document}

\title{Title goes here}
\author{Eric Andrews}

\maketitle

\section*{Hello, world!}
Here be some text for ya to read dawg. How ya like it man? You program
functionally man \cite{hughes1989functional, harbuorg}.

\printbibliography

\end{document}

% ---- EXAMPLES BELOW ----

% -- Example of figure for imagename.eps
% \begin{figure}
%     \includegraphics[scale=0.5]{imagename}
%     \caption*{Caption be here.}
% \end{figure}

% -- Example of floating figure for imagename.eps
% \begin{floatingfigure}[r]{0.49\textwidth}
%   \includegraphics[scale=0.31]{imagename}
%   \caption*{Caption be here.}
% \end{floatingfigure}

% -- Example of two figures side-by-side
% \begin{figure}[h]
%   \centering
%   \begin{subfigure}{.5\textwidth}
%     \centering
%     \includegraphics[width=.9\linewidth]{fig1.eps}
%     \caption*{Caption here}
%   \end{subfigure}%
%   \begin{subfigure}{.5\textwidth}
%     \centering
%     \includegraphics[width=.9\linewidth]{fig2.eps}
%     \caption*{Caption be here}
%   \end{subfigure}
%   \caption{Shared caption}
%   \label{fig:fighere}
% \end{figure}

% -- Remove pagination
% \thispagestyle{empty}
% \pagestyle{empty}

% -- Example of code listing
%\lstinputlisting[language=Ruby, caption={file.rb}]{./file.rb}

% -- Example of cases-environment
% $$
% f(x) =
% \begin{cases}
%     x^2 & \text{if } x > 0 \\
%     0 & \text{otherwise } \\
% \end{cases}
% $$

% -- Example of multicolumn row in tabular environment
% \multicolumn{3}{|c|}{Cell spanning three columns}
